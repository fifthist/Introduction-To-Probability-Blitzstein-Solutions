\documentclass[]{book}
\usepackage{lmodern}
\usepackage{amssymb,amsmath}
\usepackage{ifxetex,ifluatex}
\usepackage{fixltx2e} % provides \textsubscript
\ifnum 0\ifxetex 1\fi\ifluatex 1\fi=0 % if pdftex
  \usepackage[T1]{fontenc}
  \usepackage[utf8]{inputenc}
\else % if luatex or xelatex
  \ifxetex
    \usepackage{mathspec}
  \else
    \usepackage{fontspec}
  \fi
  \defaultfontfeatures{Ligatures=TeX,Scale=MatchLowercase}
\fi
% use upquote if available, for straight quotes in verbatim environments
\IfFileExists{upquote.sty}{\usepackage{upquote}}{}
% use microtype if available
\IfFileExists{microtype.sty}{%
\usepackage[]{microtype}
\UseMicrotypeSet[protrusion]{basicmath} % disable protrusion for tt fonts
}{}
\PassOptionsToPackage{hyphens}{url} % url is loaded by hyperref
\usepackage[unicode=true]{hyperref}
\hypersetup{
            pdftitle={Introduction to Probability, Second Edition},
            pdfauthor={Fifthist},
            pdfborder={0 0 0},
            breaklinks=true}
\urlstyle{same}  % don't use monospace font for urls
\usepackage{natbib}
\bibliographystyle{apalike}
\usepackage{longtable,booktabs}
% Fix footnotes in tables (requires footnote package)
\IfFileExists{footnote.sty}{\usepackage{footnote}\makesavenoteenv{long table}}{}
\usepackage{graphicx,grffile}
\makeatletter
\def\maxwidth{\ifdim\Gin@nat@width>\linewidth\linewidth\else\Gin@nat@width\fi}
\def\maxheight{\ifdim\Gin@nat@height>\textheight\textheight\else\Gin@nat@height\fi}
\makeatother
% Scale images if necessary, so that they will not overflow the page
% margins by default, and it is still possible to overwrite the defaults
% using explicit options in \includegraphics[width, height, ...]{}
\setkeys{Gin}{width=\maxwidth,height=\maxheight,keepaspectratio}
\IfFileExists{parskip.sty}{%
\usepackage{parskip}
}{% else
\setlength{\parindent}{0pt}
\setlength{\parskip}{6pt plus 2pt minus 1pt}
}
\setlength{\emergencystretch}{3em}  % prevent overfull lines
\providecommand{\tightlist}{%
  \setlength{\itemsep}{0pt}\setlength{\parskip}{0pt}}
\setcounter{secnumdepth}{5}
% Redefines (sub)paragraphs to behave more like sections
\ifx\paragraph\undefined\else
\let\oldparagraph\paragraph
\renewcommand{\paragraph}[1]{\oldparagraph{#1}\mbox{}}
\fi
\ifx\subparagraph\undefined\else
\let\oldsubparagraph\subparagraph
\renewcommand{\subparagraph}[1]{\oldsubparagraph{#1}\mbox{}}
\fi

% set default figure placement to htbp
\makeatletter
\def\fps@figure{htbp}
\makeatother

\usepackage{booktabs}
\usepackage{amsthm}
\makeatletter
\def\thm@space@setup{%
  \thm@preskip=8pt plus 2pt minus 4pt
  \thm@postskip=\thm@preskip
}
\makeatother

\title{\emph{Introduction to Probability, Second Edition}}
\author{Fifthist}
\date{2020-03-04}

\begin{document}
\maketitle

{
\setcounter{tocdepth}{1}
\tableofcontents
}
\chapter*{Preface}\label{preface}
\addcontentsline{toc}{chapter}{Preface}

This book is an unofficial solution manual for the exercises in
\emph{Introduction to Probability, Second Edition} by Joseph Blitzstein
and Jessica Hwang.

\chapter{Probability and Counting}\label{probability-and-counting}

\section{Counting}\label{counting}

\subsection{}\label{section}

\textbf{Intuition}

There are 11 slots to put letters into. We have one \(M\), four \(I\),
four \(S\), and two \(P\). Then, there are \({11 \choose 1}\) ways to
place the \(M\), \({10 \choose 4}\) ways to place \(I\),
\({6 \choose 4}\) ways to place the \(S\), and \({2 \choose 2}\) ways to
place the \(P\).

 \textbf{Solution}

\[ {11 \choose 1} \times {10 \choose 4} \times {6 \choose 4} \times {2 \choose 2} \]

\subsection{}\label{section-1}

a.~\textbf{Intuition}

If the first digit can't be \(0\) or \(1\), we are left with \(8\)
choices for the first digit. The remaining six digits can by any digits.

 \textbf{Solution}

\[ 8 \times 10^{6} \]

b.~\textbf{Intuition}

We can subtract the number of seven digits phone numbers that start with
\(911\) from the total number of phone numbers we found in the previous
part.

If a phone number starts with \(911\), it has ten choices for each of
the remaining four digits.

 \textbf{Solution}

\[ 8 \times 10^{6} - 10^{4} \]

\subsection{}\label{section-2}

a.~\textbf{Intuition}

Fred has \(10\) choices for Monday, \(9\) choices for Tuesday, \(8\)
choices for Wednesday, \(7\) choices for Thursday and \(6\) choices for
Friday.

 \textbf{Solution}

\[ 10 \times 9 \times 8 \times 7 \times 6 \]

b.~\textbf{Intution}

For the first restaurant, Fred has \(10\) choices. For all subsequent
days, Fred has \(9\) choices, sinces he doesn't want to eat at the
restaurant he ate at the previous day.

 \textbf{Solution}

\[ 10 \times 9^{4} \]

\subsection{}\label{section-3}

a.~\textbf{Intuition}

There are \({n \choose 2}\) matches in a round-robin setting. For a
given match, there are two outcomes. Irrespective of the outcome, the
next match also has two possible outcomes. Hence, we can use the
multiplication rule to count the total possible outcomes.

 \textbf{Solution}

\[ 2^{{n \choose 2}} \]

b.~\textbf{Intuition}

Since every player plays every other player exactly once, the number of
games is the number of ways to pair up \(n\) people.

 \textbf{Solution}

\[ {n \choose 2} \]

\subsection{}\label{section-4}

a.~\textbf{Intuition}

By the end of each round, half of the players participating in the round
are eliminated. So, the problem reduces to finding out how many times
can the number of players be divided by two until a single player is
left.

 \textbf{Solution}

The number of times \(N\) can be divided by two is \[\log_{2}{N}\]

b.~\textbf{Intuition}

In a given round, two opponents participate in only that one match.
Thus, the number of games in a given round is \(\frac{N_{r}}{2}\) where
\(N_{r}\) is the number of players in the \(r\)-th round.

 \textbf{Solution}

\begin{equation} 
  \begin{split}
   f(N) & = \frac{N}{2} + \frac{N}{4}  + \frac{N}{8} + \dots + \frac{N}{2^{\log_{2}{N}}}\\
   & =N \sum_{i=0}^{\log_{2}{N}} \frac{1}{2^{i}}\\
   & =N \times \frac{N-1}{N}\\
   & =N-1
  \end{split}
  \label{eq:var-beta}
  \end{equation}

c.~\textbf{Intuition}

Tournament is over when a single player is left. Hece, \(N-1\) players
need to be eliminated. As a result of a match, exactly one player is
eliminated. Hence, \(N-1\) matches are needed to eliminate \(N-1\)
people.

 \textbf{Solution}

\[ N-1 \]

\bibliography{book.bib,packages.bib}

\end{document}
