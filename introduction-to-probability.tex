\documentclass[]{book}
\usepackage{lmodern}
\usepackage{amssymb,amsmath}
\usepackage{ifxetex,ifluatex}
\usepackage{fixltx2e} % provides \textsubscript
\ifnum 0\ifxetex 1\fi\ifluatex 1\fi=0 % if pdftex
  \usepackage[T1]{fontenc}
  \usepackage[utf8]{inputenc}
\else % if luatex or xelatex
  \ifxetex
    \usepackage{mathspec}
  \else
    \usepackage{fontspec}
  \fi
  \defaultfontfeatures{Ligatures=TeX,Scale=MatchLowercase}
\fi
% use upquote if available, for straight quotes in verbatim environments
\IfFileExists{upquote.sty}{\usepackage{upquote}}{}
% use microtype if available
\IfFileExists{microtype.sty}{%
\usepackage[]{microtype}
\UseMicrotypeSet[protrusion]{basicmath} % disable protrusion for tt fonts
}{}
\PassOptionsToPackage{hyphens}{url} % url is loaded by hyperref
\usepackage[unicode=true]{hyperref}
\hypersetup{
            pdftitle={Introduction to Probability, Second Edition},
            pdfauthor={Fifthist},
            pdfborder={0 0 0},
            breaklinks=true}
\urlstyle{same}  % don't use monospace font for urls
\usepackage{natbib}
\bibliographystyle{apalike}
\usepackage{longtable,booktabs}
% Fix footnotes in tables (requires footnote package)
\IfFileExists{footnote.sty}{\usepackage{footnote}\makesavenoteenv{long table}}{}
\usepackage{graphicx,grffile}
\makeatletter
\def\maxwidth{\ifdim\Gin@nat@width>\linewidth\linewidth\else\Gin@nat@width\fi}
\def\maxheight{\ifdim\Gin@nat@height>\textheight\textheight\else\Gin@nat@height\fi}
\makeatother
% Scale images if necessary, so that they will not overflow the page
% margins by default, and it is still possible to overwrite the defaults
% using explicit options in \includegraphics[width, height, ...]{}
\setkeys{Gin}{width=\maxwidth,height=\maxheight,keepaspectratio}
\IfFileExists{parskip.sty}{%
\usepackage{parskip}
}{% else
\setlength{\parindent}{0pt}
\setlength{\parskip}{6pt plus 2pt minus 1pt}
}
\setlength{\emergencystretch}{3em}  % prevent overfull lines
\providecommand{\tightlist}{%
  \setlength{\itemsep}{0pt}\setlength{\parskip}{0pt}}
\setcounter{secnumdepth}{5}
% Redefines (sub)paragraphs to behave more like sections
\ifx\paragraph\undefined\else
\let\oldparagraph\paragraph
\renewcommand{\paragraph}[1]{\oldparagraph{#1}\mbox{}}
\fi
\ifx\subparagraph\undefined\else
\let\oldsubparagraph\subparagraph
\renewcommand{\subparagraph}[1]{\oldsubparagraph{#1}\mbox{}}
\fi

% set default figure placement to htbp
\makeatletter
\def\fps@figure{htbp}
\makeatother

\usepackage{booktabs}
\usepackage{amsthm}
\makeatletter
\def\thm@space@setup{%
  \thm@preskip=8pt plus 2pt minus 4pt
  \thm@postskip=\thm@preskip
}
\makeatother

\title{\emph{Introduction to Probability, Second Edition}}
\author{Fifthist}
\date{2020-03-05}

\begin{document}
\maketitle

{
\setcounter{tocdepth}{1}
\tableofcontents
}
\chapter*{Preface}\label{preface}
\addcontentsline{toc}{chapter}{Preface}

This book is an unofficial solution manual for the exercises in
\emph{Introduction to Probability, Second Edition} by Joseph Blitzstein
and Jessica Hwang.

\chapter{Probability and Counting}\label{probability-and-counting}

\section{Counting}\label{counting}

\subsection{}\label{section}

\textbf{Intuition}

Imagine eleven empty slots to place the letters into.

How many ways are there to place the four \(I\)-s into the slots? For
each placement of \(I\)s, can we figure out the number of ways to place
the remaining letters into the \(7\) empty slots?

 \textbf{Solution}

We have one \(M\), four \(I\)-s, four \(S\)-s, and two \(P\)-s. There
are \({11 \choose 4}\) ways to place the \(I\)-s, \({7 \choose 4}\) ways
to place \(S\)-s, \({3 \choose 2}\) ways to place the \(P\)-s, and
\({1 \choose 1}\) ways to place the \(M\).

\[ {11 \choose 4} \times {7 \choose 4} \times {3 \choose 2} \times {1 \choose 1} \]

\subsection{}\label{section-1}

a.~\textbf{Intuition}

If the first digit can't be \(0\) or \(1\), how many choices are we left
with for the first digit? For each choice of first digit, how many
choices do we have for the remaining six digits?

 \textbf{Solution}

If the first digit can't be \(0\) or \(1\), we have eight choices for
the first digit - \(2\) to \(9\). The remaining six digits can be
anything from \(0\) to \(9\). Hence, the solution is
\[ 8 \times 10^{6} \]

b.~\textbf{Intuition}

How many phone numbers start with \(911\)?

Can we use the answer from the previous part to find the desired
quantity?

 \textbf{Solution}

We can subtract the number of phone numbers that start with \(911\) from
the total number of phone numbers we found in the previous part.

If a phone number starts with \(911\), it has ten choices for each of
the remaining four digits.

\[ 8 \times 10^{6} - 10^{4} \]

\subsection{}\label{section-2}

a.~\textbf{Intuition}

How many choices of restaurants does Fred have on Monday?

Once Fred attends a restaurant on Monday, how many choices of
restaurants does he have for the remainder of the week?

 \textbf{Solution}

Fred has \(10\) choices for Monday, \(9\) choices for Tuesday, \(8\)
choices for Wednesday, \(7\) choices for Thursday and \(6\) choices for
Friday.

\[ 10 \times 9 \times 8 \times 7 \times 6 \]

b.~\textbf{Intution}

We are told that Fred will not attend a restaurant he went to the
previous day, but can he go to a restaurant he went to two or more days
ago?

 \textbf{Solution}

For the first restaurant, Fred has \(10\) choices. For all subsequent
days, Fred has \(9\) choices, since the only restriction is that he
doesn't want to eat at the restaurant he ate at the previous day.

\[ 10 \times 9^{4} \]

\subsection{}\label{section-3}

a.~\textbf{Intuition}

How many matches are there in a \emph{round-robin} tournament?

How many outcomes are possible for each match?

 \textbf{Solution}

There are \({n \choose 2}\) matches.

For a given match, there are two outcomes. Each match has two possible
outcomes. We can use the multiplication rule to count the total possible
outcomes.

\[ 2^{{n \choose 2}} \]

b.~\textbf{Intuition}

How many opponents will every player play against?

How many times will a given pair of players face each other?

 \textbf{Solution}

Since every player plays every other player exactly once, the number of
games is the number of ways to pair up \(n\) people.

\[ {n \choose 2} \]

\subsection{}\label{section-4}

a.~\textbf{Intuition}

How many players are left by the end of a round compared to the number
of players at the start of the round?

How many rounds need to pass for a single player to be left standing?

 \textbf{Solution}

By the end of each round, half of the players participating in the round
are eliminated. So, the problem reduces to finding out how many times
the number of players can be halved before a single player is left.

The number of times \(N\) can be divided by two is \[\log_{2}{N}\]

b.~\textbf{Intuition}

Suppose there are \(N_{r}\) players at the start of round \(r\). If
every player plays exactly one game, how many games will be played in
round \(r\)?

 \textbf{Solution}

The number of games in a given round is \(\frac{N_{r}}{2}\). We can sum
up these values for all the rounds.

\begin{equation} 
  \begin{split}
   f(N) & = \frac{N}{2} + \frac{N}{4}  + \frac{N}{8} + \dots + \frac{N}{2^{\log_{2}{N}}}\\
   & =N \sum_{i=0}^{\log_{2}{N}} \frac{1}{2^{i}}\\
   & =N \times \frac{N-1}{N}\\
   & =N-1
  \end{split}
  \end{equation}

c.~\textbf{Intuition}

How many players need to be eliminated before the tournament is over?

How many players are eliminated as a result of a single match?

 \textbf{Solution}

Tournament is over when a single player is left. Hece, \(N-1\) players
need to be eliminated. As a result of a match, exactly one player is
eliminated. Hence, the number of matches needed to eliminate \(N-1\)
people is

\[ N-1 \]

\subsection{}\label{section-5}

\textbf{Intuition}

How many ways can we match up twenty chess players if we don't care
about who plays with white and who plays with black pieces?

Can we use the answer from the previous part to find the desired
quantity?

 \textbf{Solution}

There are \({20 \choose 2}\) ways to pair up twenty chess players. For
each pairing, we can first let player \(A\) play with whites, then let
player \(B\) play with whites. Thus, for each of the \({20 \choose 2}\)
pairs, we have \(2\) matches for a total of

\[ {20 \choose 2} \times 2 \] matches.

\bibliography{book.bib,packages.bib}

\end{document}
