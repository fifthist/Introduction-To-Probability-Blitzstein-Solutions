\begin{enumerate}[label=(\alph*)]

\item 
Consider the simple case of $m < \frac{n}{2}$. Then, the trays don't have enough
pages to print $n$ copies. Desired probability is $0$.

On the other hand, if $m \geq n$, then desired probability is $1$, since each
tray individually has enough pages.

Now, consider the more interesting case that $\frac{n}{2} \leq m < n$.
Associate $n$ pages being taken from the trays with $n$ independent Bernoulli
trials. Sample from the first tray on success, and sample from the second tray
on failure. Thus, the assignment of trays can be modeled as a Binomial random
variable, $X \sim $ Bin$(n, p)$. As long as not too few pages are sampled from
the first tray, the remaining pages can be sampled from the second tray. What is
too few? $n-m-1$ is too few, because $n-m-1 + m < n$.

Hence,

\[ P = \begin{cases} 
      0 & m < \frac{n}{2} \\
      pbinom(m,n,p) - pbinom(n-m-1,n,p) & \frac{n}{2} \leq m < n \\
      1 & m \geq n
   \end{cases}
\]

\item Typing out the hinted program in the R language, we get that the smallest
number of papers in each tray needed to have $95$ percent confidence that there
will be enough papers to make $100$ copies is $60$.

\end{enumerate}