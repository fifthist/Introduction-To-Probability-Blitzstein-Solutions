\begin{enumerate}[label=(\alph*)]
\item $P(X \bigoplus Y) \sim$ Bern$(\frac{p}{2})$

\item If $p \neq \frac{1}{2}$, $X \bigoplus Y$ and $Y$ are not independent.
Imagine that $X=0$ is extremely unlikely. Then, knowing that $Y=0$ makes it very
likely that $X \bigoplus Y = 1$. If $p = \frac{1}{2}$, then $X \bigoplus Y$ and
$Y$ are independent.

$X \bigoplus Y$ and $X$ are
independent, since knowledge of $X$ still keeps the probability of $Y = 1$ at
$\frac{1}{2}$

\item

Let the largest element in $J$ be $m$.

\begin{flalign}
P(Y_{J}=1) & = P(X_{m}=1)P(Y_{J\setminus \{m\}}=0) + P(X_{m}=0)P(Y_{J
\setminus \{m\}}=1) \nonumber && \\
 & = \frac{1}{2}(P(Y_{J\setminus \{m\}}=0) + P(Y_{J
\setminus \{m\}}=1)) \nonumber && \\
 & = \frac{1}{2} \nonumber
\end{flalign}

Thus, $Y_{J} \sim Bern(\frac{1}{2})$
\\\\
To prove pairwise independence: Let $J$, $J'$ be two arbitrary subsets of \(\{1...n\}\). We want to show that \(P(Y_{J} = a \wedge Y_{J'} = b) = P(Y_{J}=a)P(Y_{J}=b) = 1/4\) for \(a,b \in \{0,1\}\), with the second equality coming from our knowledge that \(Y_{J} \sim Bern(1/2)\) for all \(J\). 
\\
\\
First, let us note that for all \(J, J'\) that are disjoint, \(Y_{J}, Y_{J'}\) are independent - this follows from the independence of the \(X_{i}\). 
\\\\
Now, suppose \(J, J'\) are not disjoint. Let \(A = J \cap J'\), let \(B = J \setminus A\), and let \(C = J' \setminus B\). By definition, \(A, B, C\) are disjoint. 
\\\\
Now, we have \[P(Y_{J} = a \wedge Y_{J'} = b) = P(Y_{J} = a, Y_{J'} = b | Y_{A}=1) P(Y_{A}=1) + P(Y_{J} = a, Y_{J'} = b | Y_{A}=0)P(Y_{A}=0)\]

using the LOTP. Continuing, we have

\[= P(Y_{B} = 1-a, Y_{C} = 1-b|Y_{A}=1)P(Y_{A}=1) + P(Y_{B} = a, Y_{C} = b|Y_{A}=0)P(Y_{A}=1)\]

by noting that if \(x \oplus 1 = y\), we must have \(y = 1-x\) and if \(x \oplus 1 = y\), we have \(y=x\). Continuing, we get

\[ = P(Y_{B} = 1-a)P(Y_{C}=1-b)P(Y_{A}=1) + P(Y_{B} = a)P(Y_{C}=b)P(Y_{A}=0)\]

We can remove the conditioning since \(A, B, C\) are disjoint, and therefore \(Y_{B}, Y_{C}, Y_{A}\) are all independent r.v.s. Finally, we realize that since all \(Y\) are \(Bern(1/2)\), this results in

\[ = (1/2)^3+(1/2)^3 = 1/4 = P(Y_{J} = a)P(Y_{J'}=b)\]

as desired - \(Y_{J}, Y_{J'}\) are independent for any pair \(J, J'\). \\\\

To prove that the \(Y_{J}\) are not all independent, consider the subsets \(S=\{1\}, S'=\{2\}, S''\{1,2\}\). It is clear that if \(Y_{S} = 1\) and \(Y_{S'}=1\), then \(Y_{S''} = Y_{S} \oplus Y_{S'} = 0\). However, this implies that 

\[P(\bigcap_{J \subseteq \{1..n\}}Y_{J}=1) = 0\]

i.e. it is impossible for all \(Y\) to simultaneously equal 1. However, we know that

\[\prod_{J \subseteq \{1..n\}}P(Y_{J}=1) = (1/2)^{2n-1} \neq 0\]

Thus, the \(Y_{J}\) are not independent.

\end{enumerate}
