\begin{enumerate}[label=(\alph*)]
\item The distribution is hypergeometric. We select a sample of $t$ employees
and count the number of women in the sample.

$$P(X = k) = \frac{\binom{n}{k}\binom{m}{t-k}}{\binom{n+m}{t}}$$

\item Decisions to be promoted or not are independent from employee to employee.
Thus, we are dealing with Binomial distributions.

Let $X$ be the number of women who are promoted. Then, $P(X=k) = \binom{n}{k}p^
{k}(1-p)^{n-k}$. The number of women who are not promoted is $Y=n-X$ and so is
also Binomial.

Distribution of the number of employees who are promoted is also Binomial, since
each employee is equally likely to be promoted and promotions are independent of
each other.

\item Once the total number of promotions is fixed, they are no longer
independent. For instance, if the first $t$ people are promoted, the
probability of the $t+1$-st person being promoted is $0$.

The story fits that of the hypergeometric distribution. $t$ promoted
employees
are picked and we count the number of women among them.

$$P(X = k | T = t) = \frac{\binom{n}{k}p^
{k}(1-p)^{n-k}\binom{m}{t-k}p^
{t-k}(1-p)^{m-t+k}}{\binom{n+m}{t}p^{t}(1-p)^{n+m-t}} = \frac{\binom{n}{k}
\binom{m}{t-k}}{\binom{n+m}{t}}$$
\end{enumerate}