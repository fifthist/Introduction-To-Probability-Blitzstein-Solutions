\begin{enumerate}[label=(\alph*)]
\item Let $X$ be the number of people that play the same opponent in both
rounds. Let $I_{j}$ be the indicator variable that person $j$ plays against the
same opponent twice. $P(I_{j}=1) = \frac{1}{100}$. Then, $\text{E}(X) = \sum_
{j=1}^{100}P(I_{j}=1) = 1$.

\item There is a strong dependence between trials. For instance, if we know that
the first $50$ players played the same opponent twice, then all of the players
played the same opponents twice.

\item Consider the $50$ pairs that played each other in round one. Let $I_{j}$
be the indicator variable for pair $j$ playing each other again in the second
round. $P(I_{j}=1) = \frac{1}{100}$. Then, the expected number of pairs that
play the same opponent twice is $\text{E}(X) = \frac{50}{100} = \frac{1}{2}$.

We can approximate the number of pairs that play against one another in both
rounds, $X$,  with $Z \sim \text{Poiss}(\frac{1}{2})$. $P(X = 0) \approx P(Z =
0) = e^{-\frac{1}{2}} \approx 0.6$.

$P(X = 2) \approx P(Z = 2) = \frac{(\frac{1}{2})^{2}e^{-\frac{1}{2}}}{2!}
\approx 0.08$.

This solution doesn't seem correct to me, because the trials are dependent. For
instance, in the extreme case, if we know that $49$ of the $50$ pairs play
together again, then the last pair also play again.
\end{enumerate}