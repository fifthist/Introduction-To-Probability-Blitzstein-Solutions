\begin{enumerate}[label=(\alph*)]
\item Let $C_{i}$ be the population of the $i$-th city, such that the first four
cities are in the Northern region, the next three cities are in the Eastern
region, the next two cities are in the Southern region, and the last city is in
the Western region.

Let $C$ be the population of a randomly chosen city.

Then $\text{E}(C) = \frac{1}{10}\sum_{i=1}^{10}C_{i} = 2 \text{million}.$

\item $\text{Var}(C) = \text{E}(C^{2}) - (\text{E}(C))^{2}$. $\text{E}(C^{2})$
can not be computed without the knowledge of population sizes of individual
cities.

\item $\text{Var}(C) = \frac{1}{4}(\frac{1}{4}3 \text{million} + \frac{1}{3}4
\text{million} + \frac{1}{2}5 \text{million} + 8 \text{million}) \approx 3 
\text{million}$

\item Since regions with smaller population have more cities, if a city is
randomly selected, it is more likely that the city belongs to a low population
region. On the other hand, if a region is selected uniformly at random first,
then a randomly selected city is as likely to belong to a region with a large
population as it is to belong to a region with a smaller population.
\end{enumerate}