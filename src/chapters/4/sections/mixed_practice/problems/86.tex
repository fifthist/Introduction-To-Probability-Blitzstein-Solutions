\begin{enumerate}[label=(\alph*)]
\item $P(X=x,Y=y,Z=z) = \frac{\binom{n_{A}}{x}\binom{n_{B}}{y}\binom{n_{C}}{z}}
{\binom{n}{m}}$

\item Let $I_{j}$ be the indicator variable for person $j$ in the sample being a
member of party $A$. Then, $X = \sum_{i=1}^{m}I_{i} \implies \text{E}(X) =
m\frac{n_{A}}{n}$ by symmetry.

\item Let's find \(E(X^{2})\). If we square the expression for the sum of \(X\)'s constituent indicator r.v.s, we get\\

\(E(X^{2}) = \sum_{i=1}^{m}E(I_{i}^{2}) + 2*\sum_{i < j, 1 \leq j \leq m}E(I_{i}I_{j})\)\\


Since \(I_{i}^{2} = I_{i}\), we have \(\sum_{i=1}^{m}E(I_{i}^{2}) = \frac{m*n_{A}}{n}\)\\

Additionally, for any pair \(i,j\), the r.v. \(I_{i}I_{j}\) equals 1 only when some pair of samples are both members of party A, which occurs with probability \(\frac{n_{A}(n_{A}-1)}{n(n-1)}\). There are \({m \choose 2}\) pairs \(i,j\). Therefore, the expression \(2*\sum_{i < j, 1 \leq j \leq m}E(I_{i}I_{j})\) evaluates to \(\frac{n_{A}m(m-1)(n_{A}-1)}{n(n-1)}\).\\

Finally, we have \(E(X^{2})\), so now we can write \[Var(X) = E(X^{2})-EX^{2}= \frac{m*n_{A}}{n} + \frac{n_{A}m(m-1)(n_{A}-1)}{n(n-1)} - \frac{m^{2}n_{A}^{2}}{n^{2}}\]

When $m=1$, $\text{Var}(X) = \frac{n_{A}}{n} - (\frac{n_{A}}{n})^{2} = \frac{n_
{A}}{n}(1 - \frac{n_{A}}{n}) = \frac{n_{A}}{n} \times \frac{n_{B} + n_{C}}{n}$.

When $m=n$, Var(X) = 0. This makes sense, as if the sample is the entire population, we always get the same number of members of party A in our sample (all of them), so there is no variation. 
\end{enumerate}
