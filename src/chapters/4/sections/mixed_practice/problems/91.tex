\begin{enumerate}[label=(\alph*)]
\item If $F=G$, Then, $X_{j}$ is equally likely to be in any of the $m + n$
positions in the ordered list. 
$$\text{E}(R) = \sum_{j=1}^{m}\text{E}(R_{j}) = \sum_{j=1}^{m}\frac{(m+n)
(m+n+1)}{2}\frac{1}{m+n} = m\frac{m+n+1}{2}.$$

\item $R_{j} = (\sum_{k=1}^{n}I_{Y_{k}} + \sum_{k \neq j}I_{X_{k}} + 1)$ where $I_{Y_{k}}$ are the indicator random variables for $X_{j}$ being larger than $Y_{k}$ and $I_{X_{k}}$ are the indicator random variables for $X_{j}$ being larger than $X_{k}$. Note that $E(I_{Y_{k}}) = p$ for all k since the Ys are iid, and $E(I_{X_{k}}) = 1/2$ - $X_{j}$ and $X_{k}$ are iid and never equal, so they are equally likely to be bigger or smaller than the other. Then $\text{E}(R_{j}) = np + (m-1)/2 + 1$.Thus, $\text{E}(R) = m(np + (m-1)/2 + 1)$.
\end{enumerate}
