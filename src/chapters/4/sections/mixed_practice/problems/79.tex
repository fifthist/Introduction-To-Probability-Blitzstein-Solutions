\begin{enumerate}[label=(\alph*)]
\item Let $X \sim FS(\frac{1}{m})$ be the number of guesses made by the hacker.
Then, $\text{E}(X) = m$.

\item Suppose $w_{1}, w_{2}, w_{3}, \dots, w_{m}$ is the sequence of passwords
sampled by the hacker. Since, every permutation of the $m$ words is equally
likely, the probability that the correct password is $w_{i}$ is $\frac{(m-1)!}
{m!} = \frac{1}{m}$. Then $\text{E}(X) = \frac{1}{m}\sum_{i=1}^{m}i = \frac{1}
{m}\frac{m(m+1)}{2} = \frac{m+1}{2}$.

\item Both $m$ and $\frac{m+1}{2}$ are positively sloped lines, intersecting at
$m=1$. For $m=2$, $m > \frac{m+1}{2}$. Thus, $m > \frac{m+1}{2}$ for all $m >
1$. This makes intuitive sense since when the hacker samples passwords without
replacement, the number of possible passwords reduces.

\item With replacement, $P(X = k) = (\frac{m-1}{m})^{k-1}\frac{1}{m}$ for $1
\leq k < n$ and $P(X=n) = (\frac{m-1}{m})^{n-1}\frac{1}{m} + (\frac{m-1}{m})^
{n}$.

In the case of sampling without replacement, since all orderings of the passwords
sampled by the hacker are equally likely, $P(\text{hacker samples k passwords})
= \frac{1}{m}$ for $1 \leq k < n$, and $P(\text{hacker samples n passwords}) =
\frac{1}{m} + \frac{m-n}{m}$.
\end{enumerate}