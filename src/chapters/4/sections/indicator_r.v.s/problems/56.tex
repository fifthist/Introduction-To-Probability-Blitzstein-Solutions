\begin{enumerate}[label=(\alph*)]
\item Let $I_{j}$ be the indicator variable for shots $j$ to $j+6$ being
successful. The total number of succesful, consecutive, $7$ shots is $X = \sum_
{i=1}^{n-6}I_{j}$. Then, $$\text{E}(X) = \sum_{i=1}^{n-6}\text{E}(I_{j}) = \sum_
{i=1}^{n-6}P(I_{j}=1) = \sum_{i=1}^{n-6}p^{7} = (n-6)p^{7}$$

\item Thinking of each block of $7$ shots as a single trial with probability $p^
{7}$ of success, let $Y \sim \text{Geom}(p^{7})$ be the number of
failed $7$-block shots taken until the first succesful $7$-shot block. Then, $$
\text{E}(X) = 7(1 + \text{E}(Y)) = 7 + \frac{7-7p^{7}}{p^{7}} = \frac{7p^{7} + 7
- 7p^{7}}{p^{7}} = \frac{7}{p^{7}}$$ 

Note that it is possible that a consecutive sequence of 7 shots could happen "between" blocks - for example, this way of solving the problem does not consider the scenario where shots 2 to 8 are made. Therefore, the above calculation is a "worst case scenario" that assumes the consecutive 7 made shots must always happen in the last possible block - the actual number of blocks (and therefore shots) taken to make 7 consecutive shots is strictly less than or equal to the above calculated expectation. 
\end{enumerate}
