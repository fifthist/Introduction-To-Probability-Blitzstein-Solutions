\begin{enumerate}[label=(\alph*)]
\item WLOG, let $m_{1} > m$
be the second median of $X$. Then, by the
definition of medians, $P(X \leq m) \geq \frac{1}{2}$ and $P(X \geq m_{1}) \geq 
\frac{1}{2}$. Then, $P(X \in (m,m_{1})) = 0$. If $m_{1} > m + 1$, then
there exists an $m_{2} \in (m, m_{1})$, such that $P(X = m_{2}) = 0$. This
implies that $m_{2} = 1$, since that is the only value of $X$ with
probability $0$. However, then $m < 1$, which precludes $m$ from being a
median. Thus, $m_{1}$ must be $1 + m$. Since we know $23$ to be a median of $X$,
we need to check whether $22$ or $24$ are medians of $X$. Computation via the
CDF of $X$ shows that niether $22$, nor $24$ are medians. Hence, $23$ is the
only median of $X$.

\item Let $I_{j}$ be the indicator variable for the event $X \geq j$. Notice
that the event $X = k$ (the first occurance of a birthday match happens when
there are $k$ people) implies that $I_{j}=1$ for $j \leq k$ and vice versa.
Thus, $$X = \sum_{j=1}^{366}I_{j}$$. 

Then, $$\text{E}(X) = \sum_{j=1}^{366}P(I_{j}=1) = 1 + 1 + \sum_{j=3}^{366}P(I_
{j}=1) = 2 + \sum_{j=3}^{366}p_{j}$$

\item $2 + 22.61659 = 24.61659$

\item $\text{E}(X^{2}) =
\text{E}(I_{1}^{2} + \dots + I_{366}^{2} + 2\sum_{j=2}^
{366}\sum_{i=1}^{j-1}(I_{i}I_{j}))$. Note that $I_{i}^{2} = I_{i}$ and $I_{i}I_
{j} = I_{j}$ for $i < j$. Thus,

\begin{align*}
\text{E}(X^{2}) &= \text{E}(I_{1} + \dots + I_{366} + 2\sum_{j=2}^
{366}\sum_{i=1}^{j-1}I_{j}) \\
&= 2 + \sum_{j=3}^{366}p_{j} + 2\sum_{j=2}^{366}((j-1)\text{E}(I_{j})) \\
&= 2 + \sum_{j=3}^{366}p_{j} + 2\sum_{j=2}^{366}((j-1)p_{j}) \\
&\approx 754.61659
\end{align*}

$\text{Var}(X) \approx 754.61659 - (\text{E}(X))^{2} \approx 754.61659 - 605.98 =
148.63659$.
\end{enumerate}