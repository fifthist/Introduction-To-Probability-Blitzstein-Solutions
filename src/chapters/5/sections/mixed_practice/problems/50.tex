By symmetry, and by X and Y being independent and identically distributed, the chance that X should be smaller than Y should not be different than the chance that Y should be smaller than X.
Furthermore, from independence, joint pdf, \(g(x, y) = f_X(x) f_Y(y)\)
\begin{flalign}
    P (X < Y) & = \int_{y=-\infty}^{\infty} \int_{t = -\infty}^{y} f_X(x) f_Y(y) dt dy \\
    & = \int_{y = -\infty}^{\infty} F(y) f(y) dy \\
    & = \frac{1}{2} 
\end{flalign}
When X and Y are not independent say X = Y + 1, (assume the existence of such X and Y), then, \(P(X < Y) = 0\) and \(P(Y < X) = 1\)
When X and Y are not identically distributed, say \(X \sim \text{Unif}(0, 1)\) and \(Y \sim \text{Unif}(-1, 0)\)
then, \(P(X < Y) = 1\) and \(P(Y < X) = 0\)
