(a) Let $\theta_1$, $\theta_2$ and $\theta_3$ be the angles corresponding to points A, B and C respectively.
From the statement of the problem, those angles are i.i.d. $\mathrm{Unif}(0,2\pi)$.

The angles $\theta_1$, $\theta_2$ and $\theta_3$ divide the $[0,2\pi)$ range in four successive sub-intervals.
The argument is wrong because the problem is not symmetric with respect to the three arcs, but rather with respect to the four angle sub-intervals.
The average length of each sub-interval is $2\pi/4=\pi/2$, by symmetry.
The length of the arc that contains the point (1,0) is the sum of the first and fourth sub-intervals, so it is twice as long as the other arcs on average. \\

(b)
\[\theta_1 = \text{Unif}(0, 2\pi)\]
\[\theta_2 = \text{Unif}(0, 2\pi)\]
\[\theta_3 = \text{Unif}(0, 2\pi)\]

\[L_1 = \text{min} (\theta_1, \theta_2, \theta_3)\]
CDF,
\begin{flalign*}
    F(y) & = 1 - P(\text{min}(\theta_1, \theta_2, \theta_3) > y) \\
    & = 1 - ( \frac{2\pi - y}{2\pi} )^3 \text{ , for } 0 \le y < 2\pi
\end{flalign*}
PDF,
\begin{flalign*}
    f(y) & = \frac{d}{dy} F(y) \\
    & = \frac{3}{2\pi} (1 - \frac{y}{2\pi})^2 \text{ , for } 0 \le y < 2\pi
\end{flalign*}

(c)
\begin{flalign*}
    \mathbb{E}[L] & = 2 \mathbb{E}[L_1] \\
    & = 2 \int_{0}^{2\pi} y \frac{3}{2\pi} (1 - \frac{y}{2\pi})^2 dy \\
    & = \frac{3}{\pi} \int_{0}^{2\pi} (y + \frac{y^3}{4\pi^2} - \frac{y^2}{\pi}) dy \\
    & = \frac{3}{\pi} [ \frac{4\pi^2}{2} + \frac{1}{4\pi^2} \frac{16\pi^4}{4} - \frac{1}{\pi} \frac{8\pi^3}{3}] \\
    & = \pi
\end{flalign*}

We can reach the same result from the qualitative explanation given in part (a): since $L$ is the sum of the first and fourth sub-intervals, where the length of each sub-interval is $\pi/2$ on average, $\mathbb{E}[L] = \pi/2+\pi/2 = \pi$.
