(a) 
\begin{equation}
    I_k = 
    \begin{cases}
        1 & \text{if k arrives when fun}\\
        0 & \text{if k arrives when not fun}
    \end{cases}
\end{equation}

Now \(I = I_1 + \dots + I_n \)
\begin{flalign}
    \mathbb{E}[I] & = n \mathbb{E}[I_1] \\
    & = \frac{n}{3} 
\end{flalign}

as \(P(I_k = 1) = \frac{1}{3}\)

(b)
\[P(I_1 I_2) = P(I_1) P(I_2)\]
Given that Jaime and Robert are guests 1 and 2
\[P(I_1 I_2) = P(\text{both 1 and 2 arrive when fun})\]
Out of the possible \(4!\) orderigns of Tyrion, Cersei, 1, and 2
for both 1 and 2 to arrive when fun, the following orderings are possible\\
Tyrion 1 2 Cersei
Tyrion 2 1 Cersei
Cersei 1 2 Tyrion
Cersei 2 1 Tyrion

So \(P(I_1 I_2) = \frac{1}{4!} 4 = \frac{1}{6}\)\\
\[P(I_1 I_2) \ne P(I_1) P(I_2)\]

(c)

We already know the answer. Conditioning on the event that 1 arrives when it's fun, the chances of 2 arriving when it's fun are higher than the unconditional probability of 2 arriving when it's fun.
When we have information of 1 arriving when it's fun, we know that there's someone arriving betweenTyrion and Cersei and this forces the conditional sample space to have a skew towards having Tyrion and Cersei further apart than if we have no information about 1.
This skewing of the the conditional sample space increasing chances that 2 arrives at a time when it's fun.

