a. Let \(T_{w} \sim N(w, \sigma^{2})\) be the time it takes Walter to arrive, \(T_{c} \sim N(c, 4\sigma^{2})\) be the time it takes Carl to arrive. We have \(-T_{w} \sim N(-c, \sigma^{2})\) since flipping the sign of the rv flips the sign of the expectation but does not change the variance. Then \(T_{c}-T_{w} \sim N(c-w, 5\sigma^{2})\) per the important fact given in the problem. If \(Z \sim N(0,1)\), then \(T_{c}-T_{w} = \sqrt{5}\sigma Z +c -w\).\\

For Carl to arrive first, we require \(T_{c}-T_{w} < 0\) (the time it takes Carl is less than the time it takes Walter). Let us find this probability:

\[P(T_{c}-T_{w} < 0) = P(Z < \frac{w-c}{\sigma \sqrt{5}}) = \Phi(\frac{w-c}{\sigma \sqrt{5}})\]

b. If Carl has a greater than 1/2 chance of arriving first, then \(\Phi(\frac{w-c}{\sigma \sqrt{5}}) > 1/2\). Since \(\Phi\) is an increasing function and equals 1/2 when its input is 0, this implies we need \(\frac{w-c}{\sigma \sqrt{5}} > 0\), which in turn implies \(c > w\). So, as long as Carl's car lets him be faster on average than Walter's walking, Carl has a better than 1/2 chance of arriving first. \\

c. To make it to the meeting at time, either individual needs to make sure the amount of time they take to arrive is less than \(w+10\).\\

\[P(T_{c} < w+10) = P(2\sigma Z +c < w+10) = \Phi(\frac{w+10-c}{2\sigma})\]

\[P(T_{w} < w+10) = P(\sigma Z + w < w+10) = \Phi(\frac{10}{\sigma})\]

Since \(\Phi\) is an increasing function, if we want Carl to have a greater chance than Walter to make it on time, then we require \(\frac{w+10-c}{2 \sigma} > \frac{10}{\sigma}\). This then implies that we need \(w > c+10\). 
