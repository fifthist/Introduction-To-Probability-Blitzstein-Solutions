a. We have \(A = \pi R^{2}\), so \(E(A) = \pi E(R^{2})\). We have \(E(R^{2}) = \int_{0}^{1} x^{2}*1\,dx = 1/3\), since the PDF of R is always 1. Then \(E(A) = \pi/3\). \\

We have \(\mathrm{Var}(A) = E(A^{2}) - E(A)^{2} = \pi^{2}E(R^{4}) - \pi^{2}/9\) using linearity. \(E(R^{4}) = \int_{0}^{1} x^{4}*1\,dx = 1/5\), so \(\mathrm{Var}(A) = \pi^{2}/5 - \pi^{2}/9 = 4\pi^{2}/45\)\\

b. CDF: \(P(A \le k) = P(\pi R^{2} \le k) = P(R \le \sqrt{k/\pi}) = \sqrt{k/\pi}\) for \(0<k<\pi\) using the CDF of Unif(0,1). The CDF of A is 0 for \(k \le 0\) and 1 for \(k \ge \pi\).  \\

PDF: \(\frac{d}{dk}(\sqrt{k/\pi}) = \frac{1}{2\sqrt{k\pi}}\) for \(0<k<\pi\) and 0 elsewhere.
