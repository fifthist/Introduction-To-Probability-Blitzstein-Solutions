a. See the first paragraph of part d.\\

b. \[P(B|C_{j}) = P(B \wedge C_{j})/P(C_{j}) = \frac{{48 \choose j-1}*(j-1)!*4*3*(52-j-1)!}{{48 \choose j-1}*(j-1)!*4*(52-j)!} = \frac{3}{52-j}\]

The first equality comes from Bayes' theorem. The \({48 \choose i-1}*(i-1)!\) terms come from the ways to have the first j-1 cards be non-aces. 4*3 refers to combinations of 2 adjacent aces in the numerator. The ending factorials in the numerator and denominator come from ordering the rest of the cards.
\\\\
c. We have \[P(C_{j}) = \frac{{48 \choose j-1}*(j-1)!*4*(52-j)!}{52!} \]


With the LOTP, part b, and the power of R, we can compute \[P(B) = \sum_{i=1}^{49}(P(B|C_{j})*P(C_{j}))\]

d. Argument by symmetry: Consider the events "the first card after the first ace is an ace" and "the last card after the first ace is an ace". The second event is equivalent to the last card in the deck being an ace. In addition, the two events must have the same probability, as every card drawn after the first ace is equally likely to be an ace. Therefore, the probability of the first event is 1/13.\\

For a proof using conditional probability: \\

Consider the ace of hearts and the ace of spades. The probability that the ace of hearts is the first ace to appear followed immediately by the ace of spades (call this event A) is the probability that they appear adjacent to each other in that order (call this event B) and that those two aces appear before the other two aces (call this event C).

We have 

\[P(B \wedge C) = P(C|B)*P(B)\]

Now, to compute \(P(C|B)\) consider that if the aces of hearts and of spades appear adjacent to each other in that order, we can consider them as a card "glued together". There are 3!=6 possible orderings of the glued together card and the other 2 aces - in 2 of them, the glued together card is first. So \(P(C|B) = 1/3\).\\

We also have \(P(B) = \frac{51*(50!)}{52!} = 1/52\), as there are 51 sets of two adjacent spaces the two aces could be, and the rest of the cards can be ordered in \(50!\) ways. \\

Then, we now find \(P(B \wedge C) = (1/3)*(1/52)\). \\

Now, instead of the aces of hearts and spades specifically, consider there are 12 possible pairs of aces that can be adjacent to each other. Then the total probability that the card after the first ace is another ace is 

\[12*(1/3)*(1/52) = 1/13\]
