\begin{enumerate}[label=(\alph*)]
\item Suppose $C_{1}$ contains $7$ green gummi bears and $8$ red ones, $M_{1}$
contains $1$ green gummi bear and $2$ red gummi bears, $C_{2}$ contains $5$
green gummi bears and no red gummi bears, $M_{2}$ contains $12$ green gummi
bears and $5$ red gummi bears.

The proportion of green gummi bears in $C_{1}$ is $\frac{7}{15}$, which is
larger than that of $M_{1}$, which is $\frac{1}{3}$. The proportion of green
gummi bears in $C_{2}$ is $\frac{5}{5}$, which is larger than that of $M_{2}$,
which is $\frac{12}{17}.$ However, the proportion of green gummi bears in $C_{1}
+ C_{2}$ is $\frac{12}{20}$, which is less than that of $M_{1} + M_{2}$, which
is $\frac{13}{20}.$

\item We can imagine that it is much more difficult to get a green gummi bear
out of a jar with subscript $1$ than it is out of a jar with subscript $2$.
$C$ jars have a lower overall success rate, because most of their green gummi
bears are in $C_{1}$, which is harder to sample from compared to the jars with
subscript $2$.

Let $A$ be the event that a sampled gummi bear is green. Let $B$ be the event
that the jar being sampled from is an $M$ jar. Let $C$ be the event that the
jar being sampled from has subscript $1$.

Then, by Simpson's Paradox, $P(A|B, C) < P(A|B^{c}, C)$, $P(A|B, C^{c}) < P
(A|B^{c}, C^{c})$, however, $P(A|B) > P(A|B^{c}).$
\end{enumerate}