\begin{enumerate}[label=(\alph*)]

\item $\frac{P(D|T)}{P(D^{c}|T)} = \frac{P(D)}{P(D^{c})} \frac{P(T|D)}{P(T^
{c}|D^{c})}.$

\item Suppose our population consists of $10000$ people, and only one percent of
them is afflicted with the disease. So, $100$ people have the disease and 
$9900$ people don't. Suppose the specificity and sensitivity of our test are
$95$ percent. Then, out of the $100$ people who have the disease, $95$ test 
positive and $5$ test negative, and out of the $9900$ people who do not have
the disease, $9405$ test negative and $495$ test positive.

Thus, $P(D|T) = \frac{95}{95 + 495}.$

Here, we can see why specificity matters more than sensitivity. Since, the
disease is rare, most people do not have it. Since specificity is measured as a
percentage of the population that doesn't have the disease, small changes
in specificity equate to much larger changes in the number of people than in
the case of sensitivity.

\end{enumerate}